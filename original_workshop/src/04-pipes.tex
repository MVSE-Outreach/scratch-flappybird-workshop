\section{Programming the Obstacles}
This next section builds up some scripts for the Obstacles sprite.

\subsection{Cloning}
We want the obstacles to keep coming until the game ends, but instead of making loads of different sprites, we can make lots of copies of a single sprite.

\vfill

Firstly, we need to make the original invisible:
\pic{0401-hide}

and then repeatedly create a clone:
\pic{0402-clone}

Overall, this looks like:
\pic{0403-fullclone}

\vfill



Now, we need to tell the Obstacles Sprite what to do when it is cloned:
\pic{0404-whenclone}

The obstacle needs to start off to the right of the screen, but be centered in the middle, so add in:
\pic{0405-setoffscreen}


\subsection{Randomisation}
We want to select one of the costumes by random:
\pic{0402-random}


Next we should undo the invisibility, so the clone is visible. Try to find the code piece that does that.


\subsection{Moving}
Now our cloning is set up, we need to make them move across the screen.

\vfill

Similarly to the gravity in the "Bird" sprite, we're going to put a movement piece inside a loop.

However, we want our loop to end when the sprite gets to the opposite edge of the screen, so try combining:
\pic{0406-repeatuntil}
\pic{0407-lessthan}
and
\pic{0408-xposition}
to make the loop stop when the clone reaches the left edge of the screen (where the X-coordinate is -250).

Once we don't need a clone anymore, it is really important to delete it:
\pic{0409-deleteclone}


\vfill

Now your code should look something like this:
\pic{0410-finalpipe}

\vfill

You might want to have a fiddle with the "wait" and "move" numbers here and in the bird sprite, so that the game is as easy or as hard as you wish.